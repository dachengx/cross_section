\subsection{Neutrino's interaction} \label{sec:neutrino}

Vertices and propagators in Glashow–Weinberg–Salam Theory:

Vertices:

\eqnum{
    \begin{tikzpicture}
        \begin{feynman}
            \vertex(p) at (0, 0);
            \vertex(i) at (-0.886, -0.5);
            \vertex(f) at (0.886, -0.5);
            \vertex(o) at (0, 1);
            \diagram*{
                (i) --[fermion, edge label=$l^-$, near start] (p),
                (p) --[fermion, edge label=$\nu_l$, near end] (f),
                (p) --[boson, edge label=$W^-$, near end] (o),
            };
        \end{feynman}
    \end{tikzpicture},\quad
    \frac{-ig_W}{2\sqrt{2}}\gamma^\mu(1-\gamma^5)
}

\eqnum{
    \begin{tikzpicture}
        \begin{feynman}
            \vertex(p) at (0, 0);
            \vertex(i) at (-0.886, -0.5);
            \vertex(f) at (0.886, -0.5);
            \vertex(o) at (0, 1);
            \diagram*{
                (i) --[fermion, edge label=$q_i^{-\recip{3}}$, near start] (p),
                (p) --[fermion, edge label=$q_j^{+\frac{2}{3}}$, near end] (f),
                (p) --[boson, edge label=$W^-$, near end] (o),
            };
        \end{feynman}
    \end{tikzpicture},\quad
    \frac{-ig_W}{2\sqrt{2}}\gamma^\mu(1-\gamma^5)V_{ij}
}

\eqnum{
    \begin{tikzpicture}
        \begin{feynman}
            \vertex(p) at (0, 0);
            \vertex(i) at (-0.886, -0.5);
            \vertex(f) at (0.886, -0.5);
            \vertex(o) at (0, 1);
            \diagram*{
                (i) --[fermion, edge label=$f$, near start] (p),
                (p) --[fermion, edge label=$f$, near end] (f),
                (p) --[boson, edge label=$Z$, near end] (o),
            };
        \end{feynman}
    \end{tikzpicture},\quad
    \frac{-ig_Z}{2}\gamma^\mu(c_V^f-c_A^f\gamma^5) = \frac{-ig_Z}{2}\gamma^\mu(I_3-2Q\sin^2{\theta_W}-I_3\gamma^5)
}

\begin{table}[htbp]
    \centering
    % \caption{}
    \begin{tabular}{|c c c|}
        \hline
        $f$ & $c_V$ & $c_A$ \\
        \hline
        $\nu_e,\nu_\mu,\nu_\tau$ & $\recip{2}$ & $\recip{2}$ \\
        \hline
        $e^-,\mu^-,\tau^-$ & $-\recip{2}+2\sin^2{\theta_W}$ & $-\recip{2}$ \\
        \hline
        $u,c,t$ & $\recip{2}-\frac{4}{3}\sin^2{\theta_W}$ & $\recip{2}$ \\
        \hline
        $d,s,b$ & $-\recip{2}+\frac{2}{3}\sin^2{\theta_W}$ & $-\recip{2}$ \\
        \hline
    \end{tabular}
\end{table}

Propagators:

\eqnum{
    \begin{tikzpicture}
        \begin{feynman}
            \vertex(i) at (-1, 0);
            \vertex(f) at (1, 0);
            \diagram*{
                (i) --[boson, edge label=$W^{\pm}\,\mathrm{or}\,Z$] (f),
            };
        \end{feynman}
    \end{tikzpicture},\quad
    \frac{-i}{q^2-m^2}(g_{\mu\nu}-\frac{q_\mu q_\nu}{m^2})
}

when momentum of bosons is sufficiently low, the propagator is $\frac{ig_{\mu\nu}}{m^2}$.

\subsubsection{Neutrino-electron elastic scattering}

Neutrino-electron elastic scattering, commonly known as ES,
is the important neutral current events in several neutrino detection experiments,
such as Super-Kamiokande\cite{collaboration_solar_2001} and SNO\cite{sno_collaboration_direct_2002}.

The cross section is still related to the flavor of neutrinos.
$(\nu_e,e^-)$ has a 6.5 higher cross section than $(\nu_\mu,e^-)$ and $(\nu_\tau,e^-)$.

ES provides one of the key result of neutrino flux in the solar neutrino problem.

\eqnum{
    \nu_e + e^- \rightarrow \nu_e + e^-
}

\eqnum{
    & \begin{tikzpicture}[
        arrowlabel/.style={
            /tikzfeynman/momentum/.cd,
            arrow shorten=#1,
            arrow distance=0.6,
          },
          arrowlabel/.default=0.4
        ]
        \begin{feynman}
            \vertex(p1) at (0, 0);
            \vertex(p2) at (0, 1.5);
            \vertex(i1) at (-1.5, -1.5);
            \vertex(f1) at (1.5, -1.5);
            \vertex(i2) at (-1.5, 3);
            \vertex(f2) at (1.5, 3);
            \diagram*{
                (i1) --[fermion, edge label=$\nu_e$, momentum'={[arrowlabel]$p$}] (p1),
                (p1) --[fermion, edge label=$\nu_e$, momentum'={[arrowlabel]$p^\prime$}] (f1),
                (p1) --[boson, edge label=$Z$, momentum'={[arrowlabel]$q$}] (p2),
                (i2) --[fermion, edge label=$e^-$, momentum'={[arrowlabel]$k$}] (p2),
                (p2) --[fermion, edge label=$e^-$, momentum'={[arrowlabel]$k^\prime$}] (f2),
            };
        \end{feynman}
    \end{tikzpicture} \\
    =& u^s(p)\frac{-ig_Z}{2}\gamma^\mu(\recip{2}-\recip{2}\gamma^5)\bar{u}^{s^\prime}(p^\prime)
    \frac{ig_{\mu\nu}}{m_Z^2}
    \bar{u}^r(k)\frac{-ig_Z}{2}\gamma^\nu(c_V-c_A\gamma^5)u^{r^\prime}(k^\prime)
}

The amplitude:

\eqnum{
    \label{eq:amp_ES}
    i\mathcal{M} =& -i\frac{g_Z^2}{4m_Z^2}u^s(p)\gamma^\mu(\recip{2}-\recip{2}\gamma^5)\bar{u}^{s^\prime}(p^\prime)
    \bar{u}^r(k)\gamma_\nu(c_V-c_A\gamma^5)u^{r^\prime}(k^\prime)
}

Importantly, electrons have two states of spins. But neutrinos, as massless particles, only have one state of spin,
they are always left-handed and polarized, so the spin sum of all states is:

\eqnum{
    \langle\abs{M}^2\rangle = \sum_{s}\sum_{s^\prime}\half{1}\sum_{r}\sum_{r^\prime}\abs{M(s,s^\prime\rightarrow r,r^\prime)}^2.
}

Similar to 5.2 in Peskin's book,

\eqnum{
    \label{eq:M_ES_1}
    &\langle\abs{M}^2\rangle \\
    =& \recip{2^7}(\frac{g_Z}{m_Z})^4
    \left(u^s(p)\gamma^\mu(1-\gamma^5)\bar{u}^{s^\prime}(p^\prime)u^{s^\prime}(p^\prime)(1-\gamma^5)\bar{u}^s(p)\gamma^\nu\right) \\
    &\cdot\left(\bar{u}^r(k)\gamma_\mu(c_V-c_A\gamma^5)u^{r^\prime}(k^\prime)\bar{u}^{r^\prime}(k^\prime)(c_V-c_A\gamma^5)u^r(k)\gamma_\nu\right) \\
    =& \recip{2^7}(\frac{g_Z}{m_Z})^4
    \mathrm{tr}\left[(\slashed{p}+m_\nu)\gamma^\mu(1-\gamma^5)(\slashed{p}^\prime+m_\nu)\gamma^\nu(1-\gamma^5)\right] \\
    &\cdot \mathrm{tr}\left[(\slashed{k}+m_e)\gamma_\mu(c_V-c_A\gamma^5)(\slashed{k}^\prime+m_e)\gamma_\nu(c_V-c_A\gamma^5)\right]
}

Proof a lemma, a useful general case for traces in week interaction:

\eqnum{
    &\mathrm{tr}\left[(\slashed{p}+m)\gamma^\mu(c_V-c_A\gamma^5)(\slashed{p}^\prime+m^\prime)\gamma^\nu(c_V-c_A\gamma^5)\right] \\
    =& \mathrm{tr}\left[\slashed{p}\gamma^\mu c_V\slashed{p}^\prime\gamma^\nu c_V\right]
    - \mathrm{tr}\left[\slashed{p}\gamma^\mu c_V\slashed{p}^\prime\gamma^\nu c_A\gamma^5\right] \\
    &+ \mathrm{tr}\left[\slashed{p}\gamma^\mu c_V m^\prime \gamma^\nu c_V\right]
    - \mathrm{tr}\left[\slashed{p}\gamma^\mu c_V m^\prime \gamma^\nu c_A\gamma^5\right] \\
    &- \mathrm{tr}\left[\slashed{p}\gamma^\mu c_A\gamma^5\slashed{p}^\prime\gamma^\nu c_V\right]
    + \mathrm{tr}\left[\slashed{p}\gamma^\mu c_A\gamma^5\slashed{p}^\prime\gamma^\nu c_A\gamma^5\right] \\
    &- \mathrm{tr}\left[\slashed{p}\gamma^\mu c_A\gamma^5 m^\prime \gamma^\nu c_V\right]
    + \mathrm{tr}\left[\slashed{p}\gamma^\mu c_A\gamma^5 m^\prime \gamma^\nu c_A\gamma^5\right] \\
    &+ \mathrm{tr}\left[m\gamma^\mu c_V\slashed{p}^\prime\gamma^\nu c_V\right]
    - \mathrm{tr}\left[m\gamma^\mu c_V\slashed{p}^\prime\gamma^\nu c_A\gamma^5\right] \\
    &+ \mathrm{tr}\left[m\gamma^\mu c_V m^\prime \gamma^\nu c_V\right]
    - \mathrm{tr}\left[m\gamma^\mu c_V m^\prime \gamma^\nu c_A\gamma^5\right] \\
    &- \mathrm{tr}\left[m\gamma^\mu c_A\gamma^5\slashed{p}^\prime\gamma^\nu c_V\right]
    + \mathrm{tr}\left[m\gamma^\mu c_A\gamma^5\slashed{p}^\prime\gamma^\nu c_A\gamma^5\right] \\
    &- \mathrm{tr}\left[m\gamma^\mu c_A\gamma^5 m^\prime \gamma^\nu c_V\right]
    + \mathrm{tr}\left[m\gamma^\mu c_A\gamma^5 m^\prime \gamma^\nu c_A\gamma^5\right] \\
    =& \mathrm{tr}\left[\slashed{p}\gamma^\mu c_V\slashed{p}^\prime\gamma^\nu c_V\right]
    - \mathrm{tr}\left[\slashed{p}\gamma^\mu c_V\slashed{p}^\prime\gamma^\nu c_A\gamma^5\right] \\
    &+ 0
    - 0 \\
    &- \mathrm{tr}\left[\slashed{p}\gamma^\mu c_A\gamma^5\slashed{p}^\prime\gamma^\nu c_V\right]
    + \mathrm{tr}\left[\slashed{p}\gamma^\mu c_A\gamma^5\slashed{p}^\prime\gamma^\nu c_A\gamma^5\right] \\
    &- 0
    + 0 \\
    &+ 0
    - 0 \\
    &+ \mathrm{tr}\left[m\gamma^\mu c_V m^\prime \gamma^\nu c_V\right]
    - 0 \\
    &- 0
    + 0 \\
    &- 0
    + \mathrm{tr}\left[m\gamma^\mu c_A\gamma^5 m^\prime \gamma^\nu c_A\gamma^5\right]
}

We can wait a sec here to calculate another quantity we will use later, from 5.5 in Peskin's book:

\eqnum{
    &\mathrm{tr}\left[\slashed{p}\gamma^\mu \slashed{p}^\prime\gamma^\nu \right] \\
    =& \mathrm{tr}\left[\gamma^\rho \gamma^\mu \gamma^\sigma \gamma^\nu \right]p_\rho p_\sigma^\prime \\
    =& 4(g^{\rho\mu}g^{\sigma\nu}-g^{\rho\sigma}g^{\mu\nu}+g^{\rho\nu}g^{\mu\sigma})p_\rho p_\sigma^\prime \\
    =& 4(p^\mu p^{\prime\nu} - p\cdot p^\prime g^{\mu\nu} + p^\nu p^{\prime\mu}).
}

We should noticed that $\mathrm{tr}\left[\slashed{p}\gamma^\mu c_V\slashed{p}^\prime\gamma^\nu c_V\right]$
and $\mathrm{tr}\left[\slashed{p}\gamma^\mu c_A\gamma^5\slashed{p}^\prime\gamma^\nu c_A\gamma^5\right]$ looks similar,
both with even $\gamma^\mu$ and even $\gamma^5$, actually they are the same except $c_V \rightarrow c_A$.
Because $\{\gamma^\nu,\gamma^5\}=0$.

The nasty term:

\eqnum{
    &\mathrm{tr}\left[\slashed{p}\gamma^\mu c_V\slashed{p}^\prime\gamma^\nu c_A\gamma^5\right] \\
    =& -4i c_V c_A \epsilon^{\rho\mu\sigma\nu}p_\rho p^\prime_\sigma \\
    =& \mathrm{tr}\left[\slashed{p}\gamma^\mu c_A\gamma^5\slashed{p}^\prime\gamma^\nu c_V\right]
}

Easier terms:

\eqnum{
    \mathrm{tr}\left[m\gamma^\mu c_V m^\prime \gamma^\nu c_V\right] = 4m m^\prime c_V^2 g^{\mu\nu}
}

\eqnum{
    &\mathrm{tr}\left[m\gamma^\mu c_A\gamma^5 m^\prime \gamma^\nu c_A\gamma^5\right] \\
    =& -4m m^\prime c_A^2 \mathrm{tr}\left[\gamma^\mu \gamma^\nu \gamma^5 \gamma^5\right] \\
    =& -4m m^\prime c_A^2 \mathrm{tr}\left[\gamma^\mu \gamma^\nu \right] \\
    =& -4m m^\prime c_A^2 g^{\mu\nu}
}

So:

\eqnum{
    &\mathrm{tr}\left[(\slashed{p}+m)\gamma^\mu(c_V-c_A\gamma^5)(\slashed{p}^\prime+m^\prime)\gamma^\mu(c_V-c_A\gamma^5)\right] \\
    =& 4c_V^2(p^\mu p^{\prime\nu} - p\cdot p^\prime g^{\mu\nu} + p^\nu p^{\prime\mu})
    + 4c_A^2(p^\mu p^{\prime\nu} - p\cdot p^\prime g^{\mu\nu} + p^\nu p^{\prime\mu}) \\
    & +8i c_V c_A \epsilon^{\rho\mu\sigma\nu} \\
    & +4m m^\prime (c_V^2-c_A^2) g^{\mu\nu} \\
    =& 4(c_V^2+c_A^2)(p^\mu p^{\prime\nu} + p^\nu p^{\prime\mu} - p\cdot p^\prime g^{\mu\nu})
    + 4m m^\prime (c_V^2-c_A^2) g^{\mu\nu} \\
    & +8i c_V c_A \epsilon^{\mu\nu\rho\sigma}p_\rho p^\prime_\sigma
}

Plugin the lemma back to \eqref{eq:M_ES_1}:

\eqnum{
    &\langle\abs{M}^2\rangle \\
    =& \recip{2^7}(\frac{g_Z}{m_Z})^4(
        8(p^\mu p^{\prime\nu} + p^\nu p^{\prime\mu})
        +8i \epsilon^{\mu\nu\rho\sigma}p_\rho p^\prime_\sigma
    ) \\
    &(
        4(c_V^2+c_A^2)(k_\mu k^\prime_\nu + k_\nu k^\prime_\mu - m_e^2 g_{\mu\nu})
        + 4m_e^2 (c_V^2-c_A^2) g_{\mu\nu} \\
        & +8i c_V c_A \epsilon_{\mu\nu\alpha\beta}k^\alpha k^{\prime\beta}
    ) \\
    =& ...
}

\todo[inline]{Complete calculation of $(\nu_e,e^-)$}

\subsubsection{Coherent elastic neutrino-nucleus scattering}

Coherent elastic neutrino-nucleus scattering, commonly known as CE$\nu$NS,
is an important irreducible background in the drak matter direct detection experiments,
such as XENON1T\cite{aprile_search_2021} and PandaX-4T\cite{ma_search_2023}.
The process has already been observed by COHERENT\cite{akimov_observation_2017} at the Spallation Neutron Source (SNS) at Oak Ridge National Laboratory.

To derive the cross section of such interaction,
we first apply some conventional approximation on the vertices, then dive into the stringent tree-level derivation.

\eqnum{
    & \begin{tikzpicture}[
        arrowlabel/.style={
            /tikzfeynman/momentum/.cd,
            arrow shorten=#1,
            arrow distance=0.6,
          },
          arrowlabel/.default=0.4
        ]
        \begin{feynman}
            \vertex(p1) at (0, 0);
            \vertex(p2) at (0, 1.5);
            \vertex(i1) at (-1.5, -1.5);
            \vertex(f1) at (1.5, -1.5);
            \vertex(i2) at (-1.5, 3);
            \vertex(f2) at (1.5, 3);
            \diagram*{
                (i1) --[anti fermion, edge label=$\bar{\nu_e}$, momentum'={[arrowlabel]$p$}] (p1),
                (p1) --[anti fermion, edge label=$\bar{\nu_e}$, momentum'={[arrowlabel]$p^\prime$}] (f1),
                (p1) --[boson, edge label=$Z$, momentum'={[arrowlabel]$q$}] (p2),
                (i2) --[fermion, edge label=$N$, momentum'={[arrowlabel]$k$}] (p2),
                (p2) --[fermion, edge label=$N$, momentum'={[arrowlabel]$k^\prime$}] (f2),
            };
        \end{feynman}
    \end{tikzpicture} \\
    =& \bar{v}^s(p)\frac{-ig_Z}{2}\gamma^\mu(\recip{2}-\recip{2}\gamma^5)v^{s^\prime}(p^\prime)
    \frac{ig_{\mu\nu}}{m_Z^2}
    \bar{u}^r(k)\frac{-ig_Z}{2}\gamma^\nu(c_V-c_A\gamma^5)u^{r^\prime}(k^\prime) \\
    =& \bar{v}^s(p)\frac{-ig_Z}{2}\gamma^\mu(\recip{2}-\recip{2}\gamma^5)v^{s^\prime}(p^\prime)
    \frac{ig_{\mu\nu}}{m_Z^2}
    \bar{u}^r(k)\frac{-ig_Z}{2}\gamma^\nu(I_3-2Q\sin^2{\theta_W}-I_3\gamma^5)u^{r^\prime}(k^\prime)
}

$I_3$ is the weak isospin, $Q$ is the charge. For neutron, the vertex is:

\eqnum{
    -i\frac{g_Z}{2}\gamma^\mu\left(-\recip{2}+\half{\gamma^5}\right),
}

for proton, the vertex is:

\eqnum{
    -i\frac{g_Z}{2}\gamma^\mu\left(\recip{2}-2\sin^2{\theta_W}-\half{\gamma^5}\right),
}

for neutrino, the vertex is:

\eqnum{
    -i\frac{g_Z}{2}\gamma^\mu\left(\recip{2}-\half{\gamma^5}\right).
}

$\gamma^\mu\gamma^5$ term gives the axial-vector form factor, i.e. spin-dependent form factor,
which is much smaller than the $\gamma^\mu$ term corresponding to spin-independent form
factor. We ignore the $\gamma^\mu\gamma^5$ terms for now.
\todo[inline]{Why $\gamma^\mu\gamma^5$ form factor is small?}Define the electro-weak charge as:

\eqnum{
    Q^{ew} = 2I_3-4Q\sin^2{\theta_W},
}

the vertex is just $-ig_Z Q^{ew}/4$. Summary all $Q^{ew}$ in the following table:

\begin{table}[htbp]
    \centering
    % \caption{}
    \begin{tabular}{|c c c c|}
        \hline
         & $Q$ & $I_3$ & $Q^{ew}$ \\
        \hline
        neutrino & $0$ & $+\recip{2}$ & $+1$ \\
        \hline
        neutron & $0$ & $-\recip{2}$ & $-1$ \\
        \hline
        proton & $+1$ & $+\recip{2}$ & $1-4\sin^2{\theta_W}$ \\
        \hline
    \end{tabular}
\end{table}

Rewrite the amplitude, replace $c_V-c_A\gamma^5$ by $Q^{ew}/2$ in \eqref{eq:amp_ES} and \eqref{eq:M_ES_1}:

\eqnum{
    i\mathcal{M} =& -i\frac{g_Z^2}{4m_Z^2}\bar{v}^s(p)\gamma^\mu(\recip{2}-\recip{2}\gamma^5)v^{s^\prime}(p^\prime)
    \bar{u}^r(k)\gamma_\nu\half{Q^{ew}}u^{r^\prime}(k^\prime)
}

\eqnum{
    \label{eq:M_CEvNS_1}
    &\langle\abs{M}^2\rangle \\
    =& \recip{2^9}(Q^{ew})^2(\frac{g_Z}{m_Z})^4
    \mathrm{tr}\left[(\slashed{p}+m_\nu)\gamma^\mu(1-\gamma^5)(\slashed{p}^\prime+m_\nu)\gamma^\nu(1-\gamma^5)\right] \\
    &\cdot \mathrm{tr}\left[(\slashed{k}+m_N)\gamma_\mu(\slashed{k}^\prime+m_N)\gamma_\nu\right]
}

Plugin the lemma back to \eqref{eq:M_CEvNS_1}:

\eqnum{
    &\langle\abs{M}^2\rangle \\
    =& \recip{2^9}(Q^{ew})^2(\frac{g_Z}{m_Z})^4(
        8(p^\mu p^{\prime\nu} + p^\nu p^{\prime\mu})
        +8i \epsilon^{\mu\nu\rho\sigma}p_\rho p^\prime_\sigma
    ) \\
    &\cdot (
        4(k_\mu k^\prime_\nu + k_\nu k^\prime_\mu - m_N^2 g_{\mu\nu}) + 4m_N^2 g_{\mu\nu}
    ) \\
    \approx& \recip{2^4}(Q^{ew})^2(\frac{g_Z}{m_Z})^4(
        (p^\mu p^{\prime\nu} + p^\nu p^{\prime\mu})
    ) \cdot (
        (k_\mu k^\prime_\nu + k_\nu k^\prime_\mu)
    )
}

\subsubsection{Inverse beta decay}

\clearpage
