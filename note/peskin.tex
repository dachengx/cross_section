\subsection{Derivation of equations in Peskin's book}

% TODO: add reference to the textbooks
There are many equations which are not detailedly derived in Peskin's book.

\subsubsection{2.30}

Revisit the equation 2.30:

Given

\eqnum{
    \omega_{\bf{p}} =& \sqrt{\abs{\bf{p}}+m^2} \\
    \phi(\bf{x}) =& \int\frac{\ud^3p}{(2\pi)^3}\recip{\sqrt{2\omega_{\bf{p}}}}(a_{\bf{p}}+a^\dagger_{-\bf{p}})e^{i\bf{p}\cdot\bf{x}} \\
    \pi(\bf{x}) =& \int\frac{\ud^3p}{(2\pi)^3}(-i)\sqrt{\frac{\omega_{\bf{p}}}{2}}(a_{\bf{p}}-a^\dagger_{-\bf{p}})e^{i\bf{p}\cdot\bf{x}} \\
    \left[a_{\bf{p}},a^\dagger_{\bf{p^\prime}}\right] =& (2\pi)^3\delta^{(3)}(\bf{p}-\bf{p^\prime}).
}

Thus

\eqnum{
    \left[\phi(\bf{x}),\pi(\bf{x^\prime})\right]
    =& \int\frac{\ud^3p\ud^3p^\prime}{(2\pi)^6}
    \frac{-i}{2}\sqrt{\frac{\omega_{\bf{p^\prime}}}{\omega_{\bf{p}}}}
    \left(
        \left[a^\dagger_{-\bf{p}},a_{\bf{p^\prime}}\right]-\left[a_{\bf{p}},a^\dagger_{-\bf{p^\prime}}\right]
    \right)e^{i(\bf{p}\bf{x}+\bf{p^\prime}\bf{x^\prime})} \\
    =& \int\frac{\ud^3p\ud^3p^\prime}{(2\pi)^3}
    \frac{-i}{2}\sqrt{\frac{\omega_{\bf{p^\prime}}}{\omega_{\bf{p}}}}
    \left[
        -\delta^{(3)}(-\bf{p}-\bf{p^\prime})-\delta^{(3)}(\bf{p}+\bf{p^\prime})
    \right]e^{i(\bf{p}\bf{x}+\bf{p^\prime}\bf{x^\prime})} \\
    =& i\int\frac{\ud^3p\ud^3p^\prime}{(2\pi)^3}
    \sqrt{\frac{\omega_{\bf{p^\prime}}}{\omega_{\bf{p}}}}
    \delta^{(3)}(\bf{p}+\bf{p^\prime})
    e^{i(\bf{p}\bf{x}+\bf{p^\prime}\bf{x^\prime})} \\
    =& i\int\frac{\ud^3p}{(2\pi)^3}
    e^{i\bf{p}(\bf{x}-\bf{x^\prime})} \\
    =& i\delta^{(3)}(\bf{x}-\bf{x^\prime}).
}

\subsubsection{2.31}

Revisit the equation 2.31. From equation 2.8,

\eqnum{
    H =& \int\ud^3x\mathcal{H} \\
    =& \int\ud^3x\left[\half{1}\pi^2+\half{1}(\nabla\phi)^2+\half{1}m^2\phi^2\right] \\
    =& \int\ud^3xe^{i(\bf{p}+\bf{p^\prime})\cdot\bf{x}}\int\frac{\ud^3p\ud^3p^\prime}{(2\pi)^6}
    \{
        -\frac{\sqrt{\omega_{\bf{p}}\omega_{\bf{p^\prime}}}}{4}
        (a_{\bf{p}}-a^\dagger_{-\bf{p}})(a_{\bf{p^\prime}}-a^\dagger_{-\bf{p^\prime}}) \\
        &+ \frac{-\bf{p}\cdot\bf{p^\prime}+m^2}{4\sqrt{\omega_{\bf{p}}\omega_{\bf{p^\prime}}}}
        (a_{\bf{p}}+a^\dagger_{-\bf{p}})(a_{\bf{p^\prime}}+a^\dagger_{-\bf{p^\prime}})
    \} \\
    =& \int\frac{\ud^3p\ud^3p^\prime}{(2\pi)^6}
    \{
        -\frac{\sqrt{\omega_{\bf{p}}\omega_{\bf{p^\prime}}}}{4}
        (a_{\bf{p}}-a^\dagger_{-\bf{p}})(a_{\bf{p^\prime}}-a^\dagger_{-\bf{p^\prime}}) \\
        &+ \frac{-\mathbf{p}\cdot\mathbf{p^\prime}+m^2}{4\sqrt{\omega_{\bf{p}}\omega_{\bf{p^\prime}}}}
        (a_{\bf{p}}+a^\dagger_{-\bf{p}})(a_{\bf{p^\prime}}+a^\dagger_{-\bf{p^\prime}})
    \}(2\pi)^3\delta^{(3)}(\bf{p}+\bf{p^\prime}) \\
    =& \int\frac{\ud^3p}{(2\pi)^3}
    \left[
        -\frac{\omega_{\bf{p}}}{4}
        (a_{\bf{p}}-a^\dagger_{-\bf{p}})(a_{-\bf{p}}-a^\dagger_{\bf{p}})
        + \frac{\abs{\bf{p}}+m^2}{4\omega_{\bf{p}}}
        (a_{\bf{p}}+a^\dagger_{-\bf{p}})(a_{-\bf{p}}+a^\dagger_{\bf{p}})
    \right] \\
    =& -\int\frac{\ud^3p}{(2\pi)^3}\recip{4\omega_{\bf{p}}}
    \{
        (\abs{\bf{p}}^2+m^2)\left[(a_{\bf{p}}-a^\dagger_{-\bf{p}})(a_{-\bf{p}}-a^\dagger_{\bf{p}})
        - (a_{\bf{p}}+a^\dagger_{-\bf{p}})(a_{-\bf{p}}+a^\dagger_{\bf{p}})\right]
    \} \\
    =& \int\frac{\ud^3p}{(2\pi)^3}\half{\omega_{\bf{p}}}
    \left(a_{\bf{p}}a_{\bf{p^\prime}}+a_{-\bf{p}}a_{-\bf{p^\prime}}\right) \\
    =& \int\frac{\ud^3p}{(2\pi)^3}\omega_{\bf{p}}
    \left(a_{\bf{p^\prime}}a_{\bf{p}}+\half{1}\left[a_{\bf{p}},a_{\bf{p^\prime}}\right]\right).
}

\subsubsection{2.46}

Revisit the equation 2.46:

Given

\eqnum{
    H^na_{\bf{p}} =& a_{\bf{p}}(H-E_{\bf{p}})^n.
}

Thus

\eqnum{
    e^{iHt}a_{\bf{p}} =& a_{\bf{p}}e^{i(H-E_{\bf{p}})t}.
}

Then

\eqnum{
    e^{iHt}a_{\bf{p}}e^{-iHt} =& a_{\bf{p}}e^{-iE_{\bf{p}}t}.
}

\subsubsection{2.53}

Revisit the equation 2.54:

When $x^0>y^0$,

\eqnum{
    \infint\ud p^0\recip{p^2-m^2}e^{-ip\cdot(x-y)}
    =& \int\ud p^0\recip{{p^0}^2-\bf{p}^2-m^2}e^{-ip\cdot(x-y)} \\
    =& -\half{1}(2\pi i)
    \{
        \mathop{\mathrm{Res}}(\recip{{p^0}^2-\bf{p}^2-m^2},-E_{\bf{p}}) \\
        &+ \mathop{\mathrm{Res}}(\recip{{p^0}^2-\bf{p}^2-m^2},E_{\bf{p}})
    \}e^{-ip\cdot(x-y)},
}

following the contour in the book, which goes around the poles \textcolor{red}{clockwise}. Thus

\eqnum{
    \brakaket{0}{\left[\phi(x),\phi(y)\right]}{0}
    =& \int\frac{\ud^3p}{(2\pi)^3}\recip{2E_{\bf{p}}}\left(e^{-ip\cdot(x-y)}-e^{ip\cdot(x-y)}\right) \\
    =& \int\frac{\ud^3p}{(2\pi)^3}
    \{
        \recip{2E_{\bf{p}}}e^{-ip\cdot(x-y)}\vert_{p^0=E_{\bf{p}}} \\
        &+ \recip{-2E_{\bf{p}}}e^{-ip\cdot(x-y)}\vert_{p^0=-E_{\bf{p}}}
    \} \\
    \condeq{x^0>y^0\ \ }& \int\frac{\ud^3p}{(2\pi)^3}\int\frac{\ud p^0}{2\pi i}\frac{-1}{p^2-m^2}e^{-ip\cdot(x-y)}.
}

\subsubsection{2.58}

Revisit the equation 2.58:

Given

\eqnum{
    D_R(x-y) =& \int\frac{\ud^4p}{(2\pi)^4}e^{-ip\cdot(x-y)}\tilde{D}_R(p).
}

Set $(x-y)$ as $\Delta x$, then

\eqnum{
    \tilde{D}_R(p^\prime)
    =& \int\ud^4\Delta x\int\frac{\ud^3p}{(2\pi)^3}\recip{2E_{\bf{p}}}
    \left(
        e^{-ip\cdot\Delta x} - e^{ip\cdot\Delta x}
    \right)e^{ip^\prime} \\
    =& \int\ud\Delta x^{(0)}\int\ud^3p
    \left(
        \left[\delta^(3)(\bf{p^\prime}-\bf{p})e^{-i{p^\prime}^{(0)}}-\delta^(3)(\bf{p^\prime}+\bf{p})e^{i{p^\prime}^{(0)}}\right]
    \right) \\
    =& \int\ud\Delta x^{(0)}\recip{2E_{\bf{p^\prime}}}\left(e^{-iE_{\bf{p^\prime}}\Delta x^{(0)}}-e^{iE_{\bf{p^\prime}}\Delta x^{(0)}}\right) \\
    =& \recip{2E_{\bf{p^\prime}}}\left(\recip{-iE_{\bf{p^\prime}}}-\recip{iE_{\bf{p^\prime}}}\right) \\
    =& \frac{i}{E_{\bf{p^\prime}}^2}.
}

\subsubsection{3.5}

Revisit the equation between 3.5 and 3.6, it is from equation 2.45.

\subsubsection{3.17}

Revisit the equation 3.17, David Tong's Claim 4.2.

\subsubsection{3.29}

Revisit the equation 3.29: $\TODO$

\subsubsection{3.34}

Revisit the equation 3.34: $\TODO$

\subsubsection{3.50}

Revisit the equation 3.50:

\eqnum{
    u(p)
    =& \begin{pmatrix}
        \sqrt{p\cdot\sigma}\xi \\
        \sqrt{p\cdot\bar{\sigma}}\xi
    \end{pmatrix} \\
    =& \begin{pmatrix}
        \sqrt{
            \bigl(\begin{smallmatrix}
                E-p^3 & \\
                & E+p^3
            \end{smallmatrix}\bigr)
        }\xi \\
        \sqrt{
            \bigl(\begin{smallmatrix}
                E+p^3 & \\
                & E-p^3
            \end{smallmatrix}\bigr)
        }\xi
    \end{pmatrix} \\
    =& \begin{pmatrix}
        \bigl(\begin{smallmatrix}
            \sqrt{E-p^3} & \\
            & \sqrt{E+p^3}
        \end{smallmatrix}\bigr)\xi \\
        \bigl(\begin{smallmatrix}
            \sqrt{E+p^3} & \\
            & \sqrt{E+p^3}
        \end{smallmatrix}\bigr)\xi
    \end{pmatrix} \\
    =& \begin{pmatrix}
        \left[
            \sqrt{E+p^3}\left(\half{1-\sigma^3}\right) + \sqrt{E-p^3}\left(\half{1+\sigma^3}\right)
        \right]\xi \\
        \left[
            \sqrt{E+p^3}\left(\half{1+\sigma^3}\right) + \sqrt{E-p^3}\left(\half{1-\sigma^3}\right)
        \right]\xi
    \end{pmatrix}
}

\subsubsection{3.52}

Revisit the equation 3.52:

\eqnum{
    u(p) =& \begin{pmatrix}
        \left[
            \sqrt{E+p^3}\left(\half{1-\sigma^3}\right) + \sqrt{E-p^3}\left(\half{1+\sigma^3}\right)
        \right]\begin{pmatrix}1 \\ 0\end{pmatrix} \\
        \left[
            \sqrt{E+p^3}\left(\half{1+\sigma^3}\right) + \sqrt{E-p^3}\left(\half{1-\sigma^3}\right)
        \right]\begin{pmatrix}1 \\ 0\end{pmatrix}
    \end{pmatrix} \\
    =& \begin{pmatrix}
        \left[
            \sqrt{E+p^3}\begin{pmatrix}0 & 0 \\ 0 & 1\end{pmatrix} + \sqrt{E-p^3}\begin{pmatrix}1 & 0 \\ 0 & 0\end{pmatrix}
        \right]\begin{pmatrix}1 \\ 0\end{pmatrix} \\
        \left[
            \sqrt{E+p^3}\begin{pmatrix}1 & 0 \\ 0 & 0\end{pmatrix} + \sqrt{E-p^3}\begin{pmatrix}0 & 0 \\ 0 & 1\end{pmatrix}
        \right]\begin{pmatrix}1 \\ 0\end{pmatrix}
    \end{pmatrix} \\
    =& \begin{pmatrix}
        \sqrt{E - p^3} \begin{pmatrix}1 \\ 0\end{pmatrix} \\
        \sqrt{E + p^3} \begin{pmatrix}1 \\ 0\end{pmatrix}
    \end{pmatrix}
},

which is a \textit{right-handed} particle if massless.

\subsubsection{3.65}

Revisit the equation 3.65:

\eqnum{
    u^{r\dagger}(p)v^s(p)
    =& \begin{pmatrix}
        \xi^{r\dagger}\sqrt{p\cdot\sigma}, \xi^{r\dagger}\sqrt{p\cdot\bar{\sigma}}
    \end{pmatrix}
    \cdot
    \begin{pmatrix}
        \sqrt{p\cdot\sigma}\xi^{s} \\
        -\sqrt{p\cdot\bar{\sigma}}\xi^{s}
    \end{pmatrix} \\
    =& \begin{pmatrix}
        E+p^3 & \\
        & E-p^3
    \end{pmatrix}\delta^{rs}
     - \begin{pmatrix}
        E-p^3 & \\
        & E+p^3
    \end{pmatrix}\delta^{rs} \\
    \neq& 0
}

But

\eqnum{
    u^{r\dagger}(\bf{p})v^s(-\bf{p})
    =& \begin{pmatrix}
        E-p^3 & \\
        & E+p^3
    \end{pmatrix}\delta^{rs}
     - \begin{pmatrix}
        E+p^3 & \\
        & E-p^3
    \end{pmatrix}\delta^{rs} \\
    =& 0
}

\subsubsection{3.117}

Revisit the equation 3.117:

$D_R(x-y)$ was defined at equation 2.54 and 2.55:

\eqnum{
    D_R(x-y) \equiv& \theta(x^0-y^0)\brakaket{0}{\left[\phi(x),\phi(y)\right]}{0} \\
    =& \theta(x^0-y^0)\int\frac{\ud^p}{(2\pi)^3}\recip{2E_\bf{p}}(e^{-ip\cdot(x-y)}-e^{ip\cdot (x-y)})
}

in Dirac field's context, it should be:

\eqnum{
    D_R(x-y) \equiv& \theta(x^0-y^0)\brakaket{0}{\left[\psi(x),\bar{\psi}(y)\right]}{0}.
}

Thus, also combine 3.114 and 3.115, then

\eqnum{
    & (i\slashed{\partial}+m)D_R(x-y) \\
    =& \left(i\slashed{\partial}(\theta(x^0-y^0))\right)\brakaket{0}{\left[\psi(x),\bar{\psi}(y)\right]}{0}
     + \theta(x^0-y^0)\left(i\slashed{\partial}\brakaket{0}{\left[\psi(x),\bar{\psi}(y)\right]}{0}+m\right) \\
    =& 0 + \theta(x^0-y^0)(i\slashed{\partial}+m)\brakaket{0}{\left[\psi(x),\bar{\psi}(y)\right]}{0} \\
    =& \theta(x^0-y^0)\left[
        (i\slashed{\partial}+m)\int\frac{\ud^3p}{(2\pi)^3}\recip{2E_\bf{p}}e^{-ip\cdot (x-y)}
        -(i\slashed{\partial}+m)\int\frac{\ud^3p}{(2\pi)^3}\recip{2E_\bf{p}}e^{ip\cdot (x-y)}
    \right] \\
    =& \theta(x^0-y^0)\brakaket{0}{\{\psi(x),\bar{\psi}(y)\}}{0} = S_R(x-y)
}

\subsubsection{4.52}

Revisit the equation 4.52:

$V_i$ is the value of each piece, $n_i$ is the number of piece in a possible diagram,
while the possible diagram can have infinite number of the same piece. The sum of all diagram is:

\eqnum{
    &\sum_{\substack{\text{all possible} \\ \text{connected} \\ \text{pieces}}}
    \sum_{\substack{\text{all}\{n_i\}}}\left(
        \substack{\text{value of} \\ \text{connected piece}}
    \right)\times\left(
        \prod_i \recip{n_i!}(V_i)^{n_i}
    \right) \\
    =& \left(
        \sum\text{connected}
    \right)\times
    \sum_{\substack{\text{all}\{n_i\}}}\left(
        \prod_i\recip{n_i!}(V_i)^{n_i}
    \right),
}

here "all $\{n_i\}$" is all possible ordered sets of $\{n_i, n_2, n_3, \dots\}$,
where $n_i$ can be from $0$ to $\infty$.

\eqnum{
    =& \left(
        \sum\text{connected}
    \right)\times\left(
        \sum_{n_1}\recip{n_1!}V_1^{n_1}
    \right)\left(
        \sum_{n_2}\recip{n_2!}V_2^{n_2}
    \right)\left(
        \sum_{n_3}\recip{n_3!}V_3^{n_3}
    \right)\dots \\
    =& \left(
        \sum\text{connected}
    \right)\times\prod_i\left(
        \sum_{n_i}\recip{n_i!}V_i^{n_i}
    \right),
}

here the summation includes $n_i$ from $0$ to $\infty$,

\eqnum{
    =& \left(
        \sum\text{connected}
    \right)\times\prod_i\exp(V_i) \\
    =& \left(
        \sum\text{connected}
    \right)\times\exp(\sum_iV_i)
}

\subsubsection{4.74}

Revisit the equation 4.74, it is from 4.66, 4.67, and 4.68,
where normalization in 4.66 contracts the $\abs{\phi(\bf{k})}^2$ in 4.74.

\subsubsection{4.77}

Revisit the equation 4.77:

\eqnum{
    &\int\ud\bar{k}_\mathcal{A}^z\ud\bar{k}_\mathcal{B}^z
    \delta(\bar{k}_\mathcal{A}^z + \bar{k}_\mathcal{B}^z - \sum p_f^z)
    \delta(\bar{E}_\mathcal{A} + \bar{E}_\mathcal{B} - \sum E_f) \\
    =& \int\ud\bar{k}_\mathcal{A}^z\delta(
        \sqrt{\bar{k}_\mathcal{A}^2+m_\mathcal{A}^2} + \sqrt{\bar{k}_\mathcal{B}^2+m_\mathcal{B}^2} - \sum E_f
    )\vert_{\bar{k}_\mathcal{B}^z=\sum p_f^z\bar{k}_\mathcal{A}^z} \\
    =& 1 / \abs{(
        \sqrt{\bar{k}_\mathcal{A}^2+m_\mathcal{A}^2} + \sqrt{\bar{k}_\mathcal{B}^2+m_\mathcal{B}^2} - \sum E_f
    )^\prime} \\
    =& 1 / \abs{
        \bar{k}_\mathcal{A}^z(\bar{k}_\mathcal{A}^2+m_\mathcal{A}^2)^{-1/2}
        + (\bar{k}_\mathcal{B}^z)^\prime\bar{k}_\mathcal{B}^z(\bar{k}_\mathcal{B}^2+m_\mathcal{B}^2)^{-1/2}
    } \\
    =& 1 / \abs{\frac{\bar{k}_\mathcal{A}^z}{\bar{E}_\mathcal{A}} - \frac{\bar{k}_\mathcal{B}^z}{\bar{E}_\mathcal{B}}}
}

\subsubsection{4.102}

Revisit the equation 4.102, when calculating $\brakaket{\bf{p}_1\bf{p}_2}{iT}{\bf{p}_\mathcal{A}\bf{p}_\mathcal{B}}$ based on \textit{position-space Feynman rules},
like 4.44, one of the diagram contributing to

\eqnum{
    \brakaket{\Omega}{T\{\phi(x_\mathcal{A})\phi(x_\mathcal{B})\phi(x_1)\phi(x_2)\}}{\Omega}
}

is:

\eqnum{
    & \begin{tikzpicture}
        \begin{feynman}
            \node[dot, label=above:\(z\)](z) at (0, 0);
            \node[dot, label=left:\(x_\mathcal{A}\)](i1) at (-1, -1);
            \node[dot, label=right:\(x_\mathcal{B}\)](i2) at (1, -1);
            \node[dot, label=left:\(x_1\)](f1) at (-1, 1);
            \node[dot, label=right:\(x_2\)](f2) at (1, 1);
            \node[dot, label=left:\(w\)](w) at (0.5, -0.5);
            \vertex(w0) at (1, 0);
            \diagram*{
                (i1) -- (z),
                (i2) -- (w),
                (w) -- (z),
                (z) -- (f1),
                (z) -- (f2),
                (w) --[out=0, in=-45] (w0),
                (w0) --[out=135, in=90] (w),
            };
        \end{feynman}
    \end{tikzpicture} \\
    =& \recip{2}(-i\lambda)^2
    \int\ud^4wD_F(x_\mathcal{B}-w)D_F(w-z)D_F(w-w) \\
    & \int\ud^4zD_F(x_1-z)D_F(x_2-z)D_F(x_\mathcal{A}-z)D_F(w-z) \\
    =& \recip{2}(-i\lambda)^2
    \int\ud^4w
    \int\frac{\ud^4p_\mathcal{B}}{(2\pi)^4}\frac{i}{p_\mathcal{B}^2-m^2+i\epsilon}e^{-ip_\mathcal{B}\cdot(x_\mathcal{B}-w)} \\
    & \int\frac{\ud^4p^\prime}{(2\pi)^4}\frac{i}{{p^\prime}^2-m^2+i\epsilon}e^{+ip^\prime\cdot(z-w)} \\
    & \int\frac{\ud^4k}{(2\pi)^4}\frac{i}{k^2-m^2+i\epsilon}e^{-ik\cdot(w-w)} \\
    & \cdot\int\ud^4z
    \int\frac{\ud^4p_\mathcal{A}}{(2\pi)^4}\frac{i}{p_\mathcal{A}^2-m^2+i\epsilon}e^{-ip_\mathcal{A}\cdot(x_\mathcal{A}-z)} \\
    & \int\frac{\ud^4p^\prime}{(2\pi)^4}\frac{i}{{p^\prime}^2-m^2+i\epsilon}e^{-ip^\prime\cdot(w-z)} \\
    & \int\frac{\ud^4p_1}{(2\pi)^4}\frac{i}{p_1^2-m^2+i\epsilon}e^{+ip_1\cdot(x_1-z)} \\
    & \int\frac{\ud^4p_2}{(2\pi)^4}\frac{i}{p_2^2-m^2+i\epsilon}e^{+ip_2\cdot(x_2-z)} \\
    =& \recip{2}(-i\lambda)^2
    \int\frac{\ud^4p_\mathcal{B}}{(2\pi)^4}\frac{i}{p_\mathcal{B}^2-m^2+i\epsilon} \\
    & \int\frac{\ud^4p^\prime}{(2\pi)^4}\frac{i}{{p^\prime}^2-m^2+i\epsilon} \int\frac{\ud^4k}{(2\pi)^4}\frac{i}{k^2-m^2+i\epsilon} \\
    & \cdot \int\frac{\ud^4p_\mathcal{A}}{(2\pi)^4}\frac{i}{p_\mathcal{A}^2-m^2+i\epsilon} \int\frac{\ud^4p^\prime}{(2\pi)^4}\frac{i}{{p^\prime}^2-m^2+i\epsilon} \\
    & \int\frac{\ud^4p_1}{(2\pi)^4}\frac{i}{p_1^2-m^2+i\epsilon} \int\frac{\ud^4p_2}{(2\pi)^4}\frac{i}{p_2^2-m^2+i\epsilon} \\
    & \cdot (2\pi)^4\delta^{(4)}(p_\mathcal{B}-p^\prime)(2\pi)^4\delta^{(4)}(p_\mathcal{A}+p^\prime-p_1-p_2)
}

Using LSZ reduction formula at Schwartz 6.19 and Peskin 7.42:

\eqnum{
    \brakaket{p_3\dots p_n}{S}{p_1p_2} =& \left[
        i\int\ud^4x_1e^{-ip_1x_1}(\DAlambert_1+m^2)
    \right]\dots\left[
        i\int\ud^4x_ne^{+ip_nx_n}(\DAlambert_n+m^2)
    \right] \\
    & \brakaket{\Omega}{T\{\phi(x_1)\phi(x_2)\phi(x_3)\dots\phi(x_n)\}}{\Omega},
}

where $\DAlambert = (\dpartial{}{x^\mu})^2 = \partial_t^2-\vec{\partial}_x^2$.

The above diagram's contributing to

\eqnum{
    \brakaket{\Omega}{T\{\phi(x_\mathcal{A})\phi(x_\mathcal{B})\phi(x_1)\phi(x_2)\}}{\Omega}
}

is:

\eqnum{
    & \begin{tikzpicture}
        \begin{feynman}
            \vertex (p) at (0, 0);
            \vertex(i1) at (-1, -1);
            \vertex(i2) at (1, -1);
            \vertex(f1) at (-1, 1);
            \vertex(f2) at (1, 1);
            \vertex(k) at (0.5, -0.5);
            \vertex(k0) at (1, 0);
            \diagram*{
                (i1) --[fermion, edge label=\(p_\mathcal{A}\)] (p),
                (i2) --[fermion, edge label=\(p_\mathcal{B}\)] (k),
                (k) --[fermion, edge label=\(p^\prime\)] (p),
                (p) --[fermion, edge label=\(p_1\)] (f1),
                (p) --[fermion, edge label=\(p_2\)] (f2),
                (k) --[fermion, out=0, in=-45, edge label'=\(k\)] (k0),
                (k0) --[out=135, in=90] (k),
            };
        \end{feynman}
    \end{tikzpicture} \\
    =& \half{1}
    \int\frac{\ud^4p^\prime}{(2\pi)^4}\frac{i}{{p^\prime}^2-m^2}
    \int\frac{\ud^4k}{(2\pi)^4}\frac{i}{k^2-m^2} \\
    & \times(2\pi)^4\delta^{(4)}(p_\mathcal{B}-p^\prime)(2\pi)^4\delta^{(4)}(p_\mathcal{A}+p^\prime-p_1-p_2)
}

where the external particles' Feynman propagators are cancelled. Because the $\partial_t^2$ yields 0 and $i$ in LSZ reduction formula are factored out by $\DAlambert^2+m^2$.

\subsubsection{4.107}

Revisit the equation 4.107:

Given equation 3.99 and 3.100, Dirac field operators in \textit{interaction picture} are:

\eqnum{
    \psi(x) =& \int\frac{\ud^3p}{(2\pi)^3}\recip{\sqrt{2E_\bf{p}}}\sum_s\left(
        a_\bf{p}^su^s(p)e^{-ip\cdot x} + b_\bf{p}^{s\dagger}v^s(p)e^{ip\cdot x}
    \right) \\
    \bar{\psi}(x) =& \int\frac{\ud^3p}{(2\pi)^3}\recip{\sqrt{2E_\bf{p}}}\sum_s\left(
        b_\bf{p}^s\bar{v}^s(p)e^{-ip\cdot x} + a_\bf{p}^{s\dagger}\bar{u}^s(p)e^{ip\cdot x}
    \right).
}

Decompose the Dirac field operators into positive- and negative- frequency parts:

\eqnum{
    \psi(x) =& \psi^+(x) + \psi^-(x) \\
    \bar{\psi}(x) =& \bar{\psi}^+(x) + \bar{\psi}^-(x),
}

where

\eqnum{
    \psi^+(x) =& \int\frac{\ud^3p}{(2\pi)^3}\recip{\sqrt{2E_\bf{p}}}\sum_s a_\bf{p}^su^s(p)e^{-ip\cdot x} \\
    \psi^-(x) =& \int\frac{\ud^3p}{(2\pi)^3}\recip{\sqrt{2E_\bf{p}}}\sum_s b_\bf{p}^{s\dagger}v^s(p)e^{ip\cdot x} \\
    \bar{\psi}^+(x) =& \int\frac{\ud^3p}{(2\pi)^3}\recip{\sqrt{2E_\bf{p}}}\sum_s b_\bf{p}^s\bar{v}^s(p)e^{-ip\cdot x} \\
    \bar{\psi}^-(x) =& \int\frac{\ud^3p}{(2\pi)^3}\recip{\sqrt{2E_\bf{p}}}\sum_s a_\bf{p}^{s\dagger}\bar{u}^s(p)e^{ip\cdot x}.
}

Similar to 4.33, if $x^0>y^0$,

\eqnum{
    & T\left[\psi(x)\bar{\psi}(y)\right] \\
    =& \psi^+(x)\bar{\psi}^-(y) + \bar{\psi}^-(y)\psi^+(x) + \psi^-(x)\bar{\psi}^+(y) + \psi^-(x)\bar{\psi}^+(y) + \left[\psi^+(x), \bar{\psi}^-(y)\right].
}

So, based on 3.114 and 3.121,

\eqnum{
    \brakaket{0}{T\left[\psi(x)\bar{\psi}(y)\right]}{0} =& \brakaket{0}{N\left[\psi(x)\bar{\psi}(y)\right]}{0} + \brakaket{0}{\left[\psi^+(x), \bar{\psi}^-(y)\right]}{0} \\
    \brakaket{0}{\left[\psi^+(x), \bar{\psi}^-(y)\right]}{0}
    =& \int\frac{\ud^3p}{(2\pi)^3}\recip{2E_\bf{p}}\sum_s u^s\bar{u}^s e^{-ip\cdot (x-y)} + 0 \\
    =& \brakaket{0}{\psi(x)\bar{\psi}(y)}{0} = \brakaket{0}{\{\psi^+(x), \bar{\psi}^-(y)\}}{0},
}

because in $\brakaket{0}{\bar{\psi}^-(y)\psi^+(x)}{0}$ all the $a_\bf{p}$ are at the right of all the $a^\dagger_\bf{p}$.

Similarly, when $x^0<y^0$, based on 3.115 and 3.121,

\eqnum{
    \brakaket{0}{T\left[\bar{\psi}(y)\psi(x)\right]}{0} =& \brakaket{0}{N\left[\bar{\psi}(y)\psi(x)\right]}{0} + \brakaket{0}{\left[\bar{\psi}^+(y), \psi^-(x)\right]}{0} \\
    \brakaket{0}{\left[\bar{\psi}^+(y), \psi^-(x)\right]}{0} =& -\brakaket{0}{\bar{\psi}(y)\psi(x)}{0} \\
    =& -\brakaket{0}{\{\bar{\psi}^+(y), \psi^-(x)\}}{0}.
}

So that

\eqnum{
    S_F(x-y) = \begin{cases}
        \brakaket{0}{\{\psi^+(x), \bar{\psi}^-(y)\}}{0}\ \mathrm{for}\ x^0>y^0 \\
        -\brakaket{0}{\{\bar{\psi}^+(y), \psi^-(x)\}}{0}\ \mathrm{for}\ x^0<y^0
    \end{cases}
}

and

\eqnum{
    T\left[\psi(x)\bar{\psi}(y)\right]
    =& N\left[\psi(x)\bar{\psi}(y)\right] + \wick{\c \psi(x)\bar{\c \psi}(y)} \\
    =& S_F(x-y).
}

\subsubsection{5.10}

Revisit the equation 5.10:

From 5.5, given

\eqnum{
    \mathrm{tr}(\gamma^\mu \gamma^\nu \gamma^\rho \gamma^\sigma)
    =& 4(g^{\mu\nu}g^{\rho\sigma} - g^{\mu\rho}g^{\nu\sigma} + g^{\mu\sigma}g^{\nu\rho}),
}

then

\eqnum{
    \mathrm{tr}\left[(\slashed{p}^\prime-m_e)\gamma^\mu(\slashed{p}-m_e)\gamma^\nu\right]
    =& \mathrm{tr}\left[\gamma^\rho p^\prime_\rho\gamma^\mu\gamma^\sigma p_\sigma\gamma^\nu - \gamma^\mu\gamma^\nu m_e^2\right] \\
    =& \mathrm{tr}\left[p^\prime_\rho p_\sigma - \gamma^\mu\gamma^\nu m_e^2\right] \\
    =& p^\prime_\rho p_\sigma\mathrm{tr}(\gamma^\rho\gamma^\mu\gamma^\sigma\gamma^\nu) - \mathrm{tr}(\gamma^\mu\gamma^\nu) m_e^2 \\
    =& 4p^\prime_\rho p_\sigma(g^{\rho\mu}g^{\sigma\nu} - g^{\rho\sigma}g^{\mu\nu} + g^{\rho\nu}g^{\mu\sigma}) - 4g^{\mu\nu}m_e^2 \\
    =& 4\left[p^{\prime\mu} p^\nu - g^{\mu\nu}p^\prime_\rho p^\rho + p^{\prime\nu}p^\mu - g^{\mu\nu}m_e^2\right] \\
    =& 4\left[p^{\prime\mu} p^\nu + p^{\prime\nu}p^\mu - g^{\mu\nu}(p\cdot p^\prime + m_e^2)\right].
}

So that

\eqnum{
    & \recip{4}\sum_{\mathrm{spins}}\abs{\mathcal{M}}^2 \\
    =& \frac{4e^4}{q^4}\{
        \left[(p^\prime\cdot k)(p\cdot k^\prime)+(p\cdot k)(p^\prime\cdot k^\prime)-(p\cdot p^\prime)(k\cdot k^\prime)-(p\cdot p^\prime)m_\mu^2\right] \\
        &+ \left[(p\cdot k)(p^\prime\cdot k^\prime)+(p^\prime\cdot k)(p\cdot k^\prime)-(p\cdot p^\prime)(k\cdot k^\prime)-(p\cdot p^\prime)m_\mu^2\right] \\
        &+ \left[-(p\cdot p^\prime)(k\cdot k^\prime)-(p\cdot p^\prime)(k\cdot k^\prime)+4(p\cdot p^\prime)(k\cdot k^\prime)+4(p\cdot p^\prime)m_\mu^2\right] \\
        &+ \left[-(k\cdot k^\prime)m_e^2-(k\cdot k^\prime)m_e^2+4(k\cdot k^\prime)m_e^2+4 m_\mu^2 m_e^2\right]
    \} \\
    =& \frac{8e^4}{q^4}\left[(p\cdot k)(p^\prime\cdot k^\prime) + (p\cdot k^\prime)(p^\prime\cdot k)+m_\mu^2(p\cdot p^\prime)+m_e^2(k\cdot k^\prime)+2 m_e^2 m_\mu^2\right]
}

\subsubsection{5.19}

Revisit the equation 5.19:

\eqnum{
    & \sum_{\mathrm{spins}}\abs{\bar{v}(p^\prime)\gamma^\mu(\half{1+\gamma^5})u(p)}^2 \\
    =& \mathrm{tr}\left[\slashed{p}^\prime\gamma^\mu(\half{1+\gamma^5})\slashed{p}\gamma^\nu(\half{1+\gamma^5})\right] \\
    =& \mathrm{tr}\left[\gamma^\alpha p^\prime_\alpha\gamma^\mu(\half{1+\gamma^5})\gamma^\beta p_\beta\gamma^\nu(\half{1+\gamma^5})\right]
}

Given 3.71:

\eqnum{
    \{\gamma^5,\gamma^\mu\} = 0,
}

\eqnum{
    & \mathrm{tr}\left[\gamma^\alpha p^\prime_\alpha\gamma^\mu(\half{1+\gamma^5})\gamma^\beta p_\beta\gamma^\nu(\half{1+\gamma^5})\right] \\
    =& -\mathrm{tr}\left[\gamma^\alpha p^\prime_\alpha\gamma^\mu\gamma^\beta p_\beta(\half{1+\gamma^5})\gamma^\nu(\half{1+\gamma^5})\right] \\
    =& \mathrm{tr}\left[\gamma^\alpha p^\prime_\alpha\gamma^\mu\gamma^\beta p_\beta\gamma^\nu(\half{1+\gamma^5})^2\right].
}

From 5.17:

\eqnum{
    \half{1+\gamma^5} = \begin{pmatrix}
        0 & 0 \\
        0 & 1
    \end{pmatrix},
}

\eqnum{
    \left(\half{1+\gamma^5}\right)^2 = \half{1+\gamma^5}.
}

Given 5.5, so that

\eqnum{
    & \mathrm{tr}\left[\gamma^\alpha p^\prime_\alpha\gamma^\mu\gamma^\beta p_\beta\gamma^\nu(\half{1+\gamma^5})^2\right] \\
    =& \mathrm{tr}\left[\gamma^\alpha p^\prime_\alpha\gamma^\mu\gamma^\beta p_\beta\gamma^\nu(\half{1+\gamma^5})\right] \\
    =& p^\prime_\alpha p_\beta\mathrm{tr}\left[\gamma^\alpha \gamma^\mu\gamma^\beta \gamma^\nu(\half{1+\gamma^5})\right] \\
    =& \half{1}p^\prime_\alpha p_\beta\left[\mathrm{tr}\left[\gamma^\alpha\gamma^\mu\gamma^\beta\gamma^\nu\right] + \mathrm{tr}\left[\gamma^\alpha\gamma^\mu\gamma^\beta\gamma^\nu\gamma^5\right]\right] \\
    =& \half{1}p^\prime_\alpha p_\beta\left[4(g^{\alpha\mu}g^{\beta\nu}-g^{\alpha\beta}g^{\mu\nu}+g^{\alpha\nu}g^{\mu\beta})-4i\epsilon^{\alpha\mu\beta\nu}\right] \\
    =& 2p^\prime_\alpha p_\beta(g^{\alpha\mu}g^{\beta\nu}-g^{\alpha\beta}g^{\mu\nu}+g^{\alpha\nu}g^{\mu\beta}-i\epsilon^{\alpha\mu\beta\nu}) \\
    =& 2(p^{\prime\mu}p^\nu-g^{\mu\nu}p\cdot p^\prime+p^{\prime\nu}p^\mu-i\epsilon^{\alpha\mu\beta\nu}p^\prime_\alpha p_\beta).
}

\subsubsection{5.21}

Revisit the equation 5.21:

\eqnum{
    & \abs{\mathcal{M}}^2 \\
    =& \frac{4e^4}{q^4}
    (p^{\prime\mu}p^\nu+p^{\prime\nu}p^\mu-g^{\mu\nu}p\cdot p^\prime-i\epsilon^{\alpha\mu\beta\nu}p^\prime_\alpha p_\beta)
    (k_\mu k^\prime_\nu+k_\nu k^\prime_\mu-g_{\mu\nu}k\cdot k^\prime-i\epsilon_{\rho\mu\sigma\nu}k^\rho k^{\prime\sigma}) \\
    =& \frac{4e^4}{q^4}\{
        \left[(p^\prime\cdot k)(p\cdot k^\prime)+(p\cdot k)(p^\prime\cdot k^\prime)-(p\cdot p^\prime)(k\cdot k^\prime)-ip^{\prime\mu}p^\nu\epsilon_{\rho\mu\sigma\nu}k^\rho k^{\prime\sigma}\right] \\
        &+ \left[(p\cdot k)(p^\prime\cdot k^\prime)+(p^\prime\cdot k)(p\cdot k^\prime)-(p\cdot p^\prime)(k\cdot k^\prime)-ip^{\prime\nu}p^\mu\epsilon_{\rho\mu\sigma\nu}k^\rho k^{\prime\sigma}\right] \\
        &+ \left[-(p\cdot p^\prime)(k\cdot k^\prime)-(p\cdot p^\prime)(k\cdot k^\prime)+4(p\cdot p^\prime)(k\cdot k^\prime)+ig^{\mu\nu}\epsilon_{\rho\mu\sigma\nu}k^\rho k^{\prime\sigma}\right] \\
        &+ \left[-ik_\mu k^\prime_\nu\epsilon^{\alpha\mu\beta\nu}p^\prime_\alpha p_\beta-ik_\nu k^\prime_\mu\epsilon^{\alpha\mu\beta\nu}p^\prime_\alpha p_\beta+ik\cdot k^\prime g_{\mu\nu}\epsilon^{\alpha\mu\beta\nu}p^\prime_\alpha p_\beta-\epsilon^{\alpha\mu\beta\nu}\epsilon_{\rho\mu\sigma\nu}p^\prime_\alpha p_\beta k^\rho k^{\prime\sigma}\right]
    \}
}

$g^{\mu\nu}$ is non-zero only when $\mu=\nu$, but Levi-Civita symbol $\epsilon$ is zero if any two indices are equal, so the terms analogue to $g^{\mu\nu}\epsilon_{\rho\mu\sigma\nu}k^\rho k^{\prime\sigma}$ are zero.

Additionally, it takes 3 iterations to change $\epsilon_{\rho\nu\sigma\mu}$ to $\epsilon_{\rho\mu\sigma\nu}$, so $\epsilon_{\rho\mu\sigma\nu} + \epsilon_{\rho\nu\sigma\mu}=0$,

\eqnum{
    & p^{\prime\mu}p^\nu\epsilon_{\rho\mu\sigma\nu}k^\rho k^{\prime\sigma} + p^{\prime\nu}p^\mu\epsilon_{\rho\mu\sigma\nu}k^\rho k^{\prime\sigma} \\
    =& p^{\prime\mu}p^\nu\epsilon_{\rho\mu\sigma\nu}k^\rho k^{\prime\sigma} + p^{\prime\mu}p^\nu\epsilon_{\rho\nu\sigma\mu}k^\rho k^{\prime\sigma} \\
    =& p^{\prime\mu}p^\nu(\epsilon_{\rho\mu\sigma\nu} + \epsilon_{\rho\nu\sigma\mu})k^\rho k^{\prime\sigma} = 0
}

So that,

\eqnum{
    & \abs{\mathcal{M}}^2 \\
    =& \frac{4e^4}{q^4}\left[
        2(p\cdot k)(p^\prime\cdot k^\prime)+2(p^\prime\cdot k)(p\cdot k^\prime)-\epsilon^{\alpha\mu\beta\nu}\epsilon_{\rho\mu\sigma\nu}p^\prime_\alpha p_\beta k^\rho k^{\prime\sigma}
    \right]
}

Given A.30:

\eqnum{
    \epsilon^{\alpha\beta\mu\nu}\epsilon_{\alpha\beta\rho\sigma} = -2(\delta^\mu_{\phantom{x}\rho}\delta^\nu_{\phantom{x}\sigma}-\delta^\mu_{\phantom{x}\sigma}\delta^\nu_{\phantom{x}\rho})
}

\eqnum{
    & \abs{\mathcal{M}}^2 \\
    =& \frac{8e^4}{q^4}\left[
        (p\cdot k)(p^\prime\cdot k^\prime)+(p^\prime\cdot k)(p\cdot k^\prime)+(\delta^\alpha_{\phantom{x}\rho}\delta^\beta_{\phantom{x}\sigma}-\delta^\alpha_{\phantom{x}\sigma}\delta^\beta_{\phantom{x}\rho})p^\prime_\alpha p_\beta k^\rho k^{\prime\sigma}
    \right] \\
    =& \frac{8e^4}{q^4}\left((p\cdot k)(p^\prime\cdot k^\prime)+(p^\prime\cdot k)(p\cdot k^\prime)+(p^\prime\cdot k)(p\cdot k^\prime)-(p\cdot k)(p^\prime\cdot k^\prime)\right) \\
    =& \frac{16e^4}{q^4}(p^\prime\cdot k)(p\cdot k^\prime).
}

Using the kinematical functions above 5.11,

\eqnum{
    &\phantom{=} \abs{\mathcal{M}}^2 \\
    &= \frac{16e^4}{q^4}(p^\prime\cdot k)(p\cdot k^\prime) \\
    &= \frac{e^4}{E^4}(E^2+E\abs{\bf{k}}\cos\theta)^2 \\
    &= \frac{e^4}{E^4}(E^2+E\sqrt{E^2-m_\mu^2}\cos\theta)^2 \\
    \underset{m_\mu=0}&{\approx} e^4(1+\cos\theta)^2
}

\subsubsection{5.87}

Revisit the equation 5.87:

\eqnum{
    \recip{4}\sum_{\mathrm{spins}}\abs{\mathcal{M}}^2 = 2e^4\left[
        \frac{p\cdot k^\prime}{p\cdot k} + \frac{p\cdot k}{p\cdot k^\prime} +
        2m^2(\recip{p\cdot k} - \recip{p\cdot k^\prime}) + m^4(\recip{p\cdot k} - \recip{p\cdot k^\prime})^2
    \right]
}

Given 5.81:

\begin{align}
    \recip{4}\sum_{\mathrm{spins}}\abs{\mathcal{M}}^2 &=
    \begin{aligned}[t]
        \frac{e^4}{4}g_{\mu\rho}g_{\nu\sigma}\cdot\mathrm{tr}
        \{
            &(\slashed{p}^\prime+m)
            \left[
                \frac{\gamma^\mu\slashed{k}\gamma^\nu+2\gamma^\mu p^\nu}{2p\cdot k}
                + \frac{\gamma^\nu\slashed{k}^\prime\gamma^\mu-2\gamma^\nu p^\mu}{2p\cdot k^\prime}
            \right] \notag \\
            &\cdot(\slashed{p}+m)
            \left[
                \frac{\gamma^\sigma\slashed{k}\gamma^\rho+2\gamma^\rho p^\sigma}{2p\cdot k}
                + \frac{\gamma^\rho\slashed{k}^\prime\gamma^\sigma-2\gamma^\sigma p^\rho}{2p\cdot k^\prime}
            \right]
        \} \notag \\
    \end{aligned} \\
    &\equiv \frac{e^4}{4}
    \left[
        \frac{\mathbf{I}}{(2p\cdot k)^2} + \frac{\mathbf{II}}{(2p\cdot k)(2p\cdot k^\prime)}
        + \frac{\mathbf{II}}{(2p\cdot k^\prime)(2p\cdot k)} + \frac{\mathbf{IV}}{(2p\cdot k^\prime)^2}
    \right],
\end{align}

where

\eqnum{
    \mathbf{I} =& \mathrm{tr}
    \left[
        (\slashed{p}^\prime+m)(\gamma^\mu\slashed{k}\gamma^\nu+2\gamma^\mu p^\nu)
        (\slashed{p}+m)(\gamma_\nu\slashed{k}\gamma_\mu+2\gamma_\mu p_\nu)
    \right] \\
    \mathbf{II} =& \mathrm{tr}
    \left[
        (\slashed{p}^\prime+m)(\gamma^\mu\slashed{k}\gamma^\nu+2\gamma^\mu p^\nu)
        (\slashed{p}+m)(\gamma_\mu\slashed{k}^\prime\gamma_\nu-2\gamma_\nu p_\mu)
    \right] \\
    \mathbf{III} =& \mathrm{tr}
    \left[
        (\slashed{p}^\prime+m)(\gamma^\nu\slashed{k}^\prime\gamma^\mu-2\gamma^\nu p^\mu)
        (\slashed{p}+m)(\gamma_\nu\slashed{k}\gamma_\mu+2\gamma_\mu p_\nu)
    \right] \\
    \mathbf{IV} =& \mathrm{tr}
    \left[
        (\slashed{p}^\prime+m)(\gamma^\nu\slashed{k}^\prime\gamma^\mu-2\gamma^\nu p^\mu)
        (\slashed{p}+m)(\gamma_\mu\slashed{k}^\prime\gamma_\nu-2\gamma_\nu p_\mu)
    \right].
}

We find that $\mathbf{IV}(k^\prime = -k) = \mathbf{I}(k)$,
and $\mathbf{II}=\mathbf{III}$.

For $\mathbf{I}$,
the terms that have odd number of $\gamma$(not including $\gamma^5$, see Chapter 8 Appendix of Weinberg's book for proof and reasons) is zero,
i.e. $\mathrm{tr}(\gamma^{\mu_1}\gamma^{\mu_2}\cdots\gamma^{\mu_(2N+1)})=0$(5.5, and 8.A.8 in Weinberg's book).

\eqnum{
    \mathbf{I}
    =& \mathrm{tr}
    \left[
        (\slashed{p}^\prime+m)(\gamma^\mu\slashed{k}\gamma^\nu+2\gamma^\mu p^\nu)
        (\slashed{p}+m)(\gamma_\nu\slashed{k}\gamma_\mu+2\gamma_\mu p_\nu)
    \right] \\
    =& \mathrm{tr}\left[\slashed{p}^\prime \gamma^\mu\slashed{k}\gamma^\nu \slashed{p} \gamma_\nu\slashed{k}\gamma_\mu\right]
    + \mathrm{tr}\left[2\slashed{p}^\prime \gamma^\mu\slashed{k}\gamma^\nu \slashed{p} \gamma_\mu p_\nu\right] \\
    &+ \mathrm{tr}\left[2\slashed{p}^\prime \gamma^\mu p^\nu \slashed{p} \gamma_\nu\slashed{k}\gamma_\mu\right]
    + \mathrm{tr}\left[4\slashed{p}^\prime \gamma^\mu p^\nu \slashed{p} \gamma_\mu p_\nu\right] \\
    &+ \mathrm{tr}\left[m \gamma^\mu\slashed{k}\gamma^\nu m \gamma_\nu\slashed{k}\gamma_\mu\right]
    + \mathrm{tr}\left[2m \gamma^\mu\slashed{k}\gamma^\nu m \gamma_\mu p_\nu\right] \\
    &+ \mathrm{tr}\left[2m \gamma^\mu p^\nu m \gamma_\nu\slashed{k}\gamma_\mu\right]
    + \mathrm{tr}\left[4m \gamma^\mu p^\nu m \gamma_\mu p_\nu\right]
}

Using 5.5:

\eqnum{
    \mathrm{tr}(\gamma^\mu\gamma^\nu) =& 4g^{\mu\nu} \\
    \mathrm{tr}(\gamma^\mu \gamma^\nu \gamma^\rho \gamma^\sigma)
    =& 4(g^{\mu\nu}g^{\rho\sigma} - g^{\mu\rho}g^{\nu\sigma} + g^{\mu\sigma}g^{\nu\rho}) \\
    \gamma^\mu\gamma^\nu\gamma^\rho\gamma^\sigma\gamma_mu =& -2\gamma^\sigma\gamma^\rho\gamma^\nu
}

5.9:

\eqnum{
    \gamma^\mu\gamma^\nu\gamma_\mu = -2\gamma^\nu
}

and

\eqnum{
    \slashed{p} \slashed{k} &= p_\mu \gamma^\mu \slashed{k} \\
    &= p_\mu (2k^\mu - \slashed{k}\gamma^\mu) \\
    &= 2p\cdot k - \slashed{k}\slashed{p}
},

put attention to the order $\slashed{k}\slashed{p}$ here, because $p_\mu$ is commutable with $\gamma^\mu$.

Expand the terms:

\eqnum{
    &\mathrm{tr}\left[\slashed{p}^\prime \gamma^\mu\slashed{k}\gamma^\nu \slashed{p} \gamma_\nu\slashed{k}\gamma_\mu\right] \\
    =& \mathrm{tr}\left[\slashed{p}^\prime \gamma^\mu\slashed{k}(\gamma^\nu \slashed{p} \gamma_\nu)\slashed{k}\gamma_\mu\right] \\
    =& \mathrm{tr}\left[\slashed{p}^\prime \gamma^\mu\slashed{k}(-2 \slashed{p})\slashed{k}\gamma_\mu\right] \\
    =& -2\mathrm{tr}\left[\slashed{p}^\prime (\gamma^\mu\slashed{k}\slashed{p}\slashed{k}\gamma_\mu)\right] \\
    =& -2\mathrm{tr}\left[\slashed{p}^\prime (-2\gamma^\sigma\gamma^\rho\gamma^\nu k_\sigma p_\rho k_\nu)\right] \\
    =& 4\mathrm{tr}\left[\slashed{p}^\prime \slashed{k} \slashed{p} \slashed{k}\right] \\
    =& 4\mathrm{tr}\left[\slashed{p}^\prime \slashed{k} (2p\cdot k - \slashed{k}\slashed{p})\right] \\
    =& 8p\cdot k\cdot \mathrm{tr}\left[\slashed{p}^\prime \slashed{k}\right] - 4\mathrm{tr}\left[\slashed{p}^\prime \slashed{k}\slashed{k}\slashed{p}\right] \\
    =& 8p\cdot k\cdot \mathrm{tr}\left[\gamma^\mu\gamma^\nu\right]p_\mu^\prime k_\nu - 4\mathrm{tr}(\gamma^\mu \gamma^\nu \gamma^\rho \gamma^\sigma)p_\mu^\prime k_\nu k_\rho p_\sigma \\
    =& 8p\cdot k \cdot 4g^{\mu\nu}p_\mu^\prime k_\nu - 4(g^{\mu\nu}g^{\rho\sigma} - g^{\mu\rho}g^{\nu\sigma} + g^{\mu\sigma}g^{\nu\rho})p_\mu^\prime k_\nu k_\rho p_\sigma \\
    =& 32(p\cdot k)(p^\prime\cdot k) - 4\left[(p^\prime\cdot k)(p\cdot k)-(p^\prime\cdot k)(p\cdot k)+(p^\prime\cdot p)(k\cdot k)\right] \\
    =& 32(p\cdot k)(p^\prime\cdot k),
}

\eqnum{
    &\mathrm{tr}\left[2\slashed{p}^\prime \gamma^\mu\slashed{k}\gamma^\nu \slashed{p} \gamma_\mu p_\nu\right] \\
    =& \mathrm{tr}\left[2\slashed{p}^\prime \gamma^\mu(\slashed{k}\gamma_\nu \slashed{p}) \gamma_\mu p^\nu\right] \\
    =& \mathrm{tr}\left[2\slashed{p}^\prime (-2\gamma^\sigma\gamma^\nu\gamma^\rho k_\rho p_\sigma) p^\nu\right] \\
    =& -4\mathrm{tr}\left[\slashed{p}^\prime \gamma^\sigma p_\sigma \gamma^\nu p^\nu \gamma^\rho k_\rho\right] \\
    =& -4\mathrm{tr}\left[\slashed{p}^\prime \slashed{p} \slashed{p} \slashed{k}\right] \\
    =& -4\mathrm{tr}\left[\slashed{p}^\prime p\cdot p \slashed{k}\right] \\
    =& -4m^2\mathrm{tr}\left[\slashed{p}^\prime \slashed{k}\right] \\
    =& -16m^2(p^\prime\cdot k),
}

\eqnum{
    &\mathrm{tr}\left[2\slashed{p}^\prime \gamma^\mu p^\nu \slashed{p} \gamma_\nu\slashed{k}\gamma_\mu\right] \\
    =& \mathrm{tr}\left[2\slashed{p}^\prime \gamma^\mu\slashed{k}\gamma^\nu \slashed{p} \gamma_\mu p_\nu\right] \\
    =& -16m^2(p^\prime\cdot k),
}

\eqnum{
    &\mathrm{tr}\left[4\slashed{p}^\prime \gamma^\mu p^\nu \slashed{p} \gamma_\mu p_\nu\right] \\
    =& \mathrm{tr}\left[4\slashed{p}^\prime \gamma^\mu \slashed{p} \gamma_\mu p^\nu p_\nu\right] \\
    =& 4m^2\mathrm{tr}\left[\slashed{p}^\prime \gamma^\mu \slashed{p} \gamma_\mu \right] \\
    =& 4m^2\mathrm{tr}\left[\slashed{p}^\prime (-2\slashed{p}) \right] \\
    =& -8m^2\mathrm{tr}\left[\slashed{p}^\prime \slashed{p}\right] \\
    =& -32m^2(p^\prime\cdot p),
}

\eqnum{
    &\mathrm{tr}\left[m \gamma^\mu\slashed{k}\gamma^\nu m \gamma_\nu\slashed{k}\gamma_\mu\right] \\
    =& m^2\mathrm{tr}\left[\gamma^\mu\slashed{k}(\gamma^\nu \gamma_\nu)\slashed{k}\gamma_\mu\right] \\
    =& 4m^2\mathrm{tr}\left[\gamma^\mu(\slashed{k}\slashed{k})\gamma_\mu\right] \\
    =& 4m^2 k^2\mathrm{tr}\left[\gamma^\mu\gamma_\mu\right] \\
    =& 0,
}

\eqnum{
    &\mathrm{tr}\left[2m \gamma^\mu\slashed{k}\gamma^\nu m \gamma_\mu p_\nu\right] \\
    =& 2m^2\mathrm{tr}\left[(\gamma^\mu\slashed{k}\gamma^\nu \gamma_\mu) p_\nu\right] \\
    =& 2m^2\mathrm{tr}\left[4g^{\rho\nu}k_\nu p_\nu\right] \\
    =& 32m^2(p\cdot k),
}

\eqnum{
    &\mathrm{tr}\left[2m \gamma^\mu p^\nu m \gamma_\nu\slashed{k}\gamma_\mu\right] \\
    =& \mathrm{tr}\left[2m \gamma^\mu\slashed{k}\gamma^\nu m \gamma_\mu p_\nu\right] \\
    =& 32m^2(p\cdot k),
}

\eqnum{
    &\mathrm{tr}\left[4m \gamma^\mu p^\nu m \gamma_\mu p_\nu\right] \\
    =& 4m^2\mathrm{tr}\left[\gamma^\mu p^\nu \gamma_\mu p_\nu\right] \\
    =& 4m^4\mathrm{tr}\left[\gamma^\mu \gamma_\mu\right] \\
    =& 4m^4\mathrm{tr}\left[4\times\mathbf{I}\right] \\
    =& 64m^4
}

So that

\eqnum{
    \mathbf{I} = 16(4m^4-2m^2p\cdot p^\prime + 4m^2p\cdot k - 2m^2p^\prime\cdot k + 2(p\cdot k)(p^\prime\cdot k))
}

Using Mandelstam variables in 5.83:

\eqnum{
    s &= (p+k)^2 = 2p\cdot k + m^2 = 2p^\prime\cdot k^\prime + m^2 \\
    t &= (p^\prime - p)^2 = -2p\cdot p^\prime + 2m^2 = -2k\cdot k^\prime \\
    u &= (k^\prime - p)^2 = -2k^\prime\cdot p + m^2 = -2k\cdot p^\prime + m^2
}

\eqnum{
    \mathbf{I} &= 16(4m^4-2m^2(m^2-\half{t}) + 4m^2\half{1}(s-m^2) - 2m^2\half{1}(m^2-u) + \half{1}(s-m^2)(u-m^2)) \\
    &= 16\left[m^2(4m^2 - 2m^2 + t + 2s - 2m^2 - m^2 + u) + \half{1}(s-m^2)(u-m^2)\right] \\
    &= 16\left[m^2(t + 2s + u - m^2) + \half{1}(s-m^2)(u-m^2)\right] \\
    &= 16\left[m^2(s + m^2) + \half{1}(s-m^2)(u-m^2)\right] \\
    &= 16\left[2m^4 + m^2(s-m^2) + \half{1}(s-m^2)(u-m^2)\right]
}

\subsection{Things unclear to the author in Peskin's book}

Here are some confusions in the Peskin' book.

Near equation 3.8, how to judge whether a vector is Lorentz vector?

Near equation 3.34, how to show that $\bar{\psi}\gamma^\mu\psi$ is a Lorentz vector?

Near 3.68, what is the calculation rules of $\gamma^{\mu\nu\rho}=\gamma^{\left[\mu\right.}\gamma^\nu\gamma^{\left.\rho\right]}$ and $\gamma^{\mu\nu\rho\sigma}=\gamma^{\left[\mu\right.}\gamma^\nu\gamma^\rho\gamma^{\left.\sigma\right]}$?

Near 3.113, why a right-handed spinor $v(\bf{p})$ corresponds to a positron?
In the paragraph below 5.18, the amplitude is non-zero only when the electron and positron have opposite helicity,
or equivalently, the electron and positron's spinors have same helicity.
So mathematically, the expression of spinor $u(\bf{p}),v(\bf{p})$'s helicity is defined by how they are written.
But for real particles, the helicity are defined as the orientation of observed spin, comparing to the direction of movement(which is already defined by the collision).

\clearpage
