\subsection{Derivation of equations in Peskin's book}

% TODO: add reference to the textbooks
There are many equations which are not detailedly derived in Peskin's book.

Here I first revisit the equation 2.30 and 2.31:

Given

\eqnum{
    \omega_{\bf{p}} =& \sqrt{\abs{\bf{p}}+m^2}
}

\eqnum{
    \phi(\bf{x}) =& \int\frac{\ud^3p}{(2\pi)^3}\recip{\sqrt{2\omega_{\bf{p}}}}(a_{\bf{p}}+a^\dagger_{-\bf{p}})e^{i\bf{p}\cdot\bf{x}}
}

\eqnum{
    \pi(\bf{x}) =& \int\frac{\ud^3p}{(2\pi)^3}(-i)\sqrt{\frac{\omega_{\bf{p}}}{2}}(a_{\bf{p}}-a^\dagger_{-\bf{p}})e^{i\bf{p}\cdot\bf{x}}
}

\eqnum{
    \left[a_{\bf{p}},a^\dagger_{\bf{p^\prime}}\right] =& (2\pi)^3\delta^{(3)}(\bf{p}-\bf{p^\prime})
}

\eqnum{
    \left[\phi(\bf{x}),\pi(\bf{x^\prime})\right]
    =& \int\frac{\ud^3p\ud^3p^\prime}{(2\pi)^6}
    \frac{-i}{2}\sqrt{\frac{\omega_{\bf{p^\prime}}}{\omega_{\bf{p}}}}
    \left(
        \left[a^\dagger_{-\bf{p}},a_{\bf{p^\prime}}\right]-\left[a_{\bf{p}},a^\dagger_{-\bf{p^\prime}}\right]
    \right)e^{i(\bf{p}\bf{x}+\bf{p^\prime}\bf{x^\prime})} \\
    =& \int\frac{\ud^3p\ud^3p^\prime}{(2\pi)^3}
    \frac{-i}{2}\sqrt{\frac{\omega_{\bf{p^\prime}}}{\omega_{\bf{p}}}}
    \left[
        -\delta^{(3)}(-\bf{p}-\bf{p^\prime})-\delta^{(3)}(\bf{p}+\bf{p^\prime})
    \right]e^{i(\bf{p}\bf{x}+\bf{p^\prime}\bf{x^\prime})} \\
    =& i\int\frac{\ud^3p\ud^3p^\prime}{(2\pi)^3}
    \sqrt{\frac{\omega_{\bf{p^\prime}}}{\omega_{\bf{p}}}}
    \delta^{(3)}(\bf{p}+\bf{p^\prime})
    e^{i(\bf{p}\bf{x}+\bf{p^\prime}\bf{x^\prime})} \\
    =& i\int\frac{\ud^3p}{(2\pi)^3}
    e^{i\bf{p}(\bf{x}-\bf{x^\prime})} \\
    =& i\delta^{(3)}(\bf{x}-\bf{x^\prime})
}

from equation 2.8,

\eqnum{
    H =& \int\ud^3x\mathcal{H} \\
    =& \int\ud^3x\left[\half{1}\pi^2+\half{1}(\nabla\phi)^2+\half{1}m^2\phi^2\right] \\
    =& \int\ud^3xe^{i(\bf{p}+\bf{p^\prime})\cdot\bf{x}}\int\frac{\ud^3p\ud^3p^\prime}{(2\pi)^6}
    \{
        -\frac{\sqrt{\omega_{\bf{p}}\omega_{\bf{p^\prime}}}}{4}
        (a_{\bf{p}}-a^\dagger_{-\bf{p}})(a_{\bf{p^\prime}}-a^\dagger_{-\bf{p^\prime}}) \\
        &+ \frac{-\bf{p}\cdot\bf{p^\prime}+m^2}{4\sqrt{\omega_{\bf{p}}\omega_{\bf{p^\prime}}}}
        (a_{\bf{p}}+a^\dagger_{-\bf{p}})(a_{\bf{p^\prime}}+a^\dagger_{-\bf{p^\prime}})
    \} \\
    =& \int\frac{\ud^3p\ud^3p^\prime}{(2\pi)^6}
    \{
        -\frac{\sqrt{\omega_{\bf{p}}\omega_{\bf{p^\prime}}}}{4}
        (a_{\bf{p}}-a^\dagger_{-\bf{p}})(a_{\bf{p^\prime}}-a^\dagger_{-\bf{p^\prime}}) \\
        &+ \frac{-\mathbf{p}\cdot\mathbf{p^\prime}+m^2}{4\sqrt{\omega_{\bf{p}}\omega_{\bf{p^\prime}}}}
        (a_{\bf{p}}+a^\dagger_{-\bf{p}})(a_{\bf{p^\prime}}+a^\dagger_{-\bf{p^\prime}})
    \}(2\pi)^3\delta^{(3)}(\bf{p}+\bf{p^\prime}) \\
    =& \int\frac{\ud^3p}{(2\pi)^3}
    \left[
        -\frac{\omega_{\bf{p}}}{4}
        (a_{\bf{p}}-a^\dagger_{-\bf{p}})(a_{-\bf{p}}-a^\dagger_{\bf{p}})
        + \frac{\abs{\bf{p}}+m^2}{4\omega_{\bf{p}}}
        (a_{\bf{p}}+a^\dagger_{-\bf{p}})(a_{-\bf{p}}+a^\dagger_{\bf{p}})
    \right] \\
    =& -\int\frac{\ud^3p}{(2\pi)^3}\recip{4\omega_{\bf{p}}}
    \{
        (\abs{\bf{p}}^2+m^2)\left[(a_{\bf{p}}-a^\dagger_{-\bf{p}})(a_{-\bf{p}}-a^\dagger_{\bf{p}})
        - (a_{\bf{p}}+a^\dagger_{-\bf{p}})(a_{-\bf{p}}+a^\dagger_{\bf{p}})\right]
    \} \\
    =& \int\frac{\ud^3p}{(2\pi)^3}\half{\omega_{\bf{p}}}
    \left(a_{\bf{p}}a_{\bf{p^\prime}}+a_{-\bf{p}}a_{-\bf{p^\prime}}\right) \\
    =& \int\frac{\ud^3p}{(2\pi)^3}\omega_{\bf{p}}
    \left(a_{\bf{p^\prime}}a_{\bf{p}}+\half{1}\left[a_{\bf{p}},a_{\bf{p^\prime}}\right]\right)
}

Revisit the equation 2.46:

Given

\eqnum{
    H^na_{\bf{p}} =& a_{\bf{p}}(H-E_{\bf{p}})^n
}

So

\eqnum{
    e^{iHt}a_{\bf{p}} =& a_{\bf{p}}e^{i(H-E_{\bf{p}})t}
}

Then

\eqnum{
    e^{iHt}a_{\bf{p}}e^{-iHt} =& a_{\bf{p}}e^{-iE_{\bf{p}}t}
}

Revisit the equation 2.54:

When $x^0>y^0$,

\eqnum{
    \infint\ud p^0\recip{p^2-m^2}e^{-ip\cdot(x-y)}
    =& \int\ud p^0\recip{{p^0}^2-\bf{p}^2-m^2}e^{-ip\cdot(x-y)} \\
    =& -\half{1}(2\pi i)
    \{
        \mathop{\mathrm{Res}}(\recip{{p^0}^2-\bf{p}^2-m^2},-E_{\bf{p}}) \\
        &+ \mathop{\mathrm{Res}}(\recip{{p^0}^2-\bf{p}^2-m^2},E_{\bf{p}})
    \}e^{-ip\cdot(x-y)}
}

following the contour in the book, which goes around the poles \textcolor{red}{clockwise}.

\eqnum{
    \brakaket{0}{\left[\phi(x),\phi(y)\right]}{0}
    =& \int\frac{\ud^3p}{(2\pi)^3}\recip{2E_{\bf{p}}}\left(e^{-ip\cdot(x-y)}-e^{ip\cdot(x-y)}\right) \\
    =& \int\frac{\ud^3p}{(2\pi)^3}
    \{
        \recip{2E_{\bf{p}}}e^{-ip\cdot(x-y)}\vert_{p^0=E_{\bf{p}}} \\
        &+ \recip{-2E_{\bf{p}}}e^{-ip\cdot(x-y)}\vert_{p^0=-E_{\bf{p}}}
    \} \\
    \condeq{x^0>y^0\ \ }& \int\frac{\ud^3p}{(2\pi)^3}\int\frac{\ud p^0}{2\pi i}\frac{-1}{p^2-m^2}e^{-ip\cdot(x-y)}
}

Revisit the equation 2.58:

Given

\eqnum{
    D_R(x-y) =& \int\frac{\ud^4p}{(2\pi)^4}e^{-ip\cdot(x-y)}\tilde{D}_R(p)
}

Set $(x-y)$ as $\Delta x$

\eqnum{
    \tilde{D}_R(p^\prime)
    =& \int\ud^4\Delta x\int\frac{\ud^3p}{(2\pi)^3}\recip{2E_{\bf{p}}}
    \left(
        e^{-ip\cdot\Delta x} - e^{ip\cdot\Delta x}
    \right)e^{ip^\prime} \\
    =& \int\ud\Delta x^{(0)}\int\ud^3p
    \left(
        \left[\delta^(3)(\bf{p^\prime}-\bf{p})e^{-i{p^\prime}^{(0)}}-\delta^(3)(\bf{p^\prime}+\bf{p})e^{i{p^\prime}^{(0)}}\right]
    \right) \\
    =& \int\ud\Delta x^{(0)}\recip{2E_{\bf{p^\prime}}}\left(e^{-iE_{\bf{p^\prime}}\Delta x^{(0)}}-e^{iE_{\bf{p^\prime}}\Delta x^{(0)}}\right) \\
    =& \recip{2E_{\bf{p^\prime}}}\left(\recip{-iE_{\bf{p^\prime}}}-\recip{iE_{\bf{p^\prime}}}\right) \\
    =& \frac{i}{E_{\bf{p^\prime}}^2}
}

Revisit the formula between 3.5 and 3.6, it is from 2.45. 

Revisit 3.17, David Tong's Claim 4.2.

Revisit 3.29, 3.34  % TODO

Revisit 3.50:

\eqnum{
    u(p)
    =& \begin{pmatrix}
        \sqrt{p\cdot\sigma}\xi \\
        \sqrt{p\cdot\bar{\sigma}}\xi
    \end{pmatrix} \\
    =& \begin{pmatrix}
        \sqrt{
            \bigl(\begin{smallmatrix}
                E+p^3 & \\
                & E-p^3
            \end{smallmatrix}\bigr)
        }\xi \\
        \sqrt{
            \bigl(\begin{smallmatrix}
                E-p^3 & \\
                & E+p^3
            \end{smallmatrix}\bigr)
        }\xi
    \end{pmatrix} \\
    =& \begin{pmatrix}
        \bigl(\begin{smallmatrix}
            \sqrt{E+p^3} & \\
            & \sqrt{E-p^3}
        \end{smallmatrix}\bigr)\xi \\
        \bigl(\begin{smallmatrix}
            \sqrt{E-p^3} & \\
            & \sqrt{E+p^3}
        \end{smallmatrix}\bigr)\xi
    \end{pmatrix} \\
    =& \begin{pmatrix}
        \left[
            \sqrt{E+p^3}\left(\half{1-\sigma^3}\right) + \sqrt{E-p^3}\left(\half{1+\sigma^3}\right)
        \right]\xi \\
        \left[
            \sqrt{E-p^3}\left(\half{1+\sigma^3}\right) + \sqrt{E+p^3}\left(\half{1-\sigma^3}\right)
        \right]\xi
    \end{pmatrix}
}

Revisit 3.65:

\eqnum{
    u^{r\dagger}(p)v^s(p)
    =& \begin{pmatrix}
        \xi^{r\dagger}\sqrt{p\cdot\sigma}, \xi^{r\dagger}\sqrt{p\cdot\bar{\sigma}}
    \end{pmatrix}
    \cdot
    \begin{pmatrix}
        \sqrt{p\cdot\sigma}\xi^{s} \\
        -\sqrt{p\cdot\bar{\sigma}}\xi^{s}
    \end{pmatrix} \\
    =& \begin{pmatrix}
        E+p^3 & \\
        & E-p^3
    \end{pmatrix}\delta^{rs}
     - \begin{pmatrix}
        E-p^3 & \\
        & E+p^3
    \end{pmatrix}\delta^{rs} \\
    \neq& 0
}

\eqnum{
    u^{r\dagger}(\bf{p})v^s(-\bf{p})
    =& \begin{pmatrix}
        E-p^3 & \\
        & E+p^3
    \end{pmatrix}\delta^{rs}
     - \begin{pmatrix}
        E+p^3 & \\
        & E-p^3
    \end{pmatrix}\delta^{rs} \\
    =& 0
}
